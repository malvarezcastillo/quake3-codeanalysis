\documentclass[a4paper,12pt]{report}
\usepackage[spanish,activeacute]{babel}
\usepackage[utf8]{inputenc}
\usepackage{graphicx}
\usepackage{eurosym}
\usepackage{url}
\renewcommand\thesection{\arabic{section}}



\begin{document}
	
	\pagestyle{empty}
	\begin{titlepage}
		\begin{center}
			\includegraphics[scale=0.15]{images/fic01.png} \\ 
			\vspace{2cm} \includegraphics[scale=.4]{images/udc.pdf} \\
			
			
			\vspace{2.5cm}
			
			\textbf{\Large Diseño Software en un Quake III Arena}\\
			\vspace{0.5cm}
			\large{Máster Universitario en Ingeniería Informática}\\
		\end{center}
		
		\vspace{7.2cm}
		\begin{flushright}
			\noindent Elena M. Delamano Freije
			
			\noindent Martín Álvarez Castillo
			
		\end{flushright}
	\end{titlepage}
	\clearpage
	
	
	\tableofcontents
	\clearpage
	
	
	\section{Introducción}
	Quake III Arena, a partir de ahora referido como Q3, es un videojuego de disparos en primera persona (FPS) que fue lanzado en el año 1999 por \textit{id Software}. Este anticipado lanzamiento, al igual que el resto de juegos de \textit{id}, revolucionó el género de los FPS, tanto a nivel de diseño \,---\,el cual no se comentará en este documento excepto donde sea relevante\,---\,, como a nivel de tecnologías e implementación de motor gráfico de tiempo real en el ámbito de los videojuegos. \cite{quake3}\\
	
	El nuevo motor que se desarrolló para crear Q3 fue bautizado como \textit{id tech 3}, cuando nos refiramos a Q3 realmente nos estaremos refiriendo a la versión de \textit{id tech 3} empleada para el desarrollo de Q3. Para desarrollos comerciales \textit{id Software} ofreció una licencia de su nuevo motor a empresas de terceros. Una de las muchas empresas que licenció \textit{id tech 3} fue Activision, para el desarrollo de la primera edición de \textit{Call of Duty}. La licencia del motor permitía la modificación del mismo, y a día de hoy la familia de juegos de la franquicia de Call of Duty todos usan una versión modificada cuya raíz es \textit{id tech 3}. \cite{idtech3}\\
	
	Asimismo, siguiendo la filosofía de "compartir y colaborar para avanzar la tecnología" del programador líder John Carmack, \textit{id Software} liberó todo el código fuente de Q3 bajo la licencia GPL-2.0 \cite{sourcecode}. La liberación de este código provocó que el juego fuera portado a muchas nuevas arquitecturas y, al tener dependencias con licencias abiertas, permitió que los fans hicieran versiones mejoradas del juego completamente retrocompatibles con el contenido pasado, añadiendo funcionalidades nuevas y arreglando bugs conocidos. Una de estas implementaciones de software libre muy popular es \textit{ioquake3} \cite{ioquake3} \\
	
	goals, actors, status, releases \cite{example}\\
	
	\section{Metodología de Desarrollo y Herramientas}
	https://github.com/id-Software/Quake-III-Arena\\
	
	\section{Arquitectura de Software: Patrones y Antipatrones}
	patterns and anti-patterns\\
		
	\section{Diseño de Software}
	design patterns and anti-patterns\\
	
	\section{Calidad del Software}
	metrics, documentation, testing and CI, etc\\
	
	\section{Estado de la accesibilidad en el proyecto}
	
	\section{Conclusiones}
			
	
	
	
	
	
	\begin{thebibliography}{9}
		
		\bibitem{example}Google \emph{Ejemplo Bibliografía}. URL: \url{http://www.google.com}\\
		\bibitem{quake3}Quake III Arena \emph{Quake Wikia}. URL: \url{http://quake.wikia.com/wiki/Quake_III_Arena}\\
		\bibitem{idtech3} id tech 3 \emph{Giant Bomb}. URL: \url{https://www.giantbomb.com/id-tech-3/3015-1918/}\\
		\bibitem{sourcecode} Quake III Arena Source Code \emph{Github}. URL: \url{https://github.com/id-Software/Quake-III-Arena}\\
		\bibitem{ioquake3} ioquake3 \emph{ioquake3}. URL: \url{https://ioquake3.org/}
		
		
	\end{thebibliography}
	
	
\end{document}
